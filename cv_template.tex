\documentclass[a4paper,10pt]{article}

\usepackage[T1]{fontenc}
\usepackage[utf8]{inputenc}
\usepackage[english]{babel}
\usepackage[skins,breakable]{tcolorbox}

\usepackage{geometry}
\geometry{
    textwidth=190mm,
    textheight=267mm,
	inner=10mm,
	outer=10mm,
	top=10mm,
	bottom=20mm
}
\usepackage{fancyhdr}
\usepackage{graphicx}
\usepackage{fontawesome}
\usepackage{hyperref}
\hypersetup{
	colorlinks = true,
	urlcolor = cyan,
	linkcolor = black,
}

%%%%%%%%%%%%%%%%%%%% - Usage Changes - %%%%%%%%%%%%%%%%%%%%
\newcommand{\professional}{Name}
\newcommand{\age}{Age}
\newcommand{\address}{Address}
\newcommand{\phone}{Phone number}
\newcommand{\email}{email@email.com}
\renewcommand{\date}{\today}
\newcommand{\about}{
    Text about yourself.
	\vspace{3\baselineskip}
}
%%%%%%%%%%%%%%%%% - End of Usage Changes - %%%%%%%%%%%%%%%%

\setlength{\fboxrule}{2pt}
\setlength{\fboxsep}{0pt}

\fancyhead{} % Header: reset
\fancyfoot{} % Footer: reset
\fancyfoot[C]{Page \thepage \ of \pageref{lastPage}} % Footer: center
\fancyfoot[R]{Made with \LaTeX} % Footer: right
\fancyfoot[L]{Updated in \date} % Footer: left
\renewcommand{\headrulewidth}{0pt} % Header line thickness
\renewcommand{\footrulewidth}{0.4pt} % Footer line thickness
\pagestyle{fancy} % Sets the page style

\newcommand{\createSection}[4][0]{
    \begin{tcolorbox}[
        blanker,
        breakable,
        title=\begin{minipage}{0.16\linewidth}\large{\textbf{#2}}\vspace{-#3\baselineskip}\end{minipage},
        coltitle=black,
        leftupper=0.21\linewidth,
    ]
        #4
		\ifnum0#1>0 { \hrule {\ } } \fi
    \end{tcolorbox}
}

\begin{document}

	\noindent
	\fbox{
	\hspace*{-3.35\fboxrule}
	\begin{minipage}{0.3\linewidth}
		\includegraphics[width=\linewidth]{img/profile.png}
	\end{minipage}
	\hspace*{-3.35\fboxrule}
	}
	\hfill
	\begin{minipage}{0.65\linewidth}
		\Huge{\bf \professional, \age}\\\vspace{-1.75\baselineskip}

		\noindent\rule{\textwidth}{1.5pt} {\ }\\\vspace{-1.8\baselineskip}

		\large{
		\faMapMarker \ \address \\
		\begin{minipage}{0.5\linewidth}
			\faWhatsapp \ \phone
		\end{minipage}
		\begin{minipage}{0.5\linewidth}
			\faEnvelope \ \email
		\end{minipage}
		\begin{minipage}{0.5\linewidth}
			\faLinkedinSquare \ \href{https://www.linkedin.com}{\texttt{/user-linkedin}}
		\end{minipage}
		\begin{minipage}{0.5\linewidth}
			\faGithub \ \href{https://github.com}{\texttt{/user-github}}
		\end{minipage}
		\faLink \ \href{https://www.overleaf.com/}{\texttt{other-social-media}}\\
		\vfill
		\textbf{About}:\about
		}
	\end{minipage}
	\vspace{\baselineskip}

%%%%%%%%%%%%%%%%%%%%%% - CV's Body - %%%%%%%%%%%%%%%%%%%%%%
    \createSection[1]{Education}{2}{
		\textit{Course}, Institute, Place \hfill year - year \\
	}

	\createSection[1]{Experience}{2}{
	    \textit{Role}, Institute, Place \hfill month year - currently \\
	    \textbf{Activities}: Activities\\
	}

    \createSection[1]{Programming languages and tools}{4}{
        \large{\bf
			\begin{minipage}{0.33\linewidth}
				Language\\
				Language\\
				\vspace{\baselineskip}
			\end{minipage}
			\begin{minipage}{0.33\linewidth}
				Tool\\
				\vspace{2\baselineskip}
			\end{minipage}
			\begin{minipage}{0.33\linewidth}
				\vspace{3\baselineskip}
			\end{minipage}
		}
    }

	\createSection{Languages}{2}{
	    \large{
			\begin{minipage}{0.5\linewidth}
				\textbf{Language}: level \\
			\end{minipage}
			\begin{minipage}{0.5\linewidth}
				\textbf{Language}: level \\
			\end{minipage}
		}
	}
%%%%%%%%%%%%%%%%%%% - End of CV's Body - %%%%%%%%%%%%%%%%%%
    \label{lastPage}
\end{document}
